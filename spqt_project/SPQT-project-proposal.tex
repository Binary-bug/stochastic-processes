\documentclass[a4paper,12pt]{scrartcl}

\usepackage{graphicx}
\usepackage{%
	algorithmic,%
	amsmath,%
	amssymb,%
	amsthm,%
	%booktabs,%
	caption,%
	%fixltx2e,%
	hyperref,%
	%mathpazo,%
	%siunitx,%
	%subcaption,%
	%xspace,%
	pgf,%
	pgfgantt,%
	pgfplots,%
	tikz,%
	url,%
	xcolor,%
	tabto,%
	} 
%\usepackage{color}

\date{}
\usepackage[left=18mm,right=18mm,top=20mm,bottom=20mm]{geometry}
\usepackage[utf8]{inputenc} %-- pour utiliser des accents en français, ou autres
\usepackage[mathscr]{eucal}
%\usepackage[round]{natbib}
%\usepackage[dvipsnames]{xcolor}%
\usepgflibrary{shapes}
\usetikzlibrary{%
	arrows,positioning, shapes.symbols,shapes.callouts,patterns
}
\newcommand{\sS}{\mathscr{S}}
\newcommand{\ie}{i.e.}
\newcommand{\eg}{e.g.}
\newcommand{\reffig}[1]{Figure~\ref{#1}}
\newcommand{\refsec}[1]{Section~\ref{#1}}
\renewcommand{\le}{\leqslant}
\renewcommand{\ge}{\geqslant}
\newcommand{\EQ}[1]{\begin{equation*}#1\end{equation*}}
\setcapindent{1em} %-- for captions of Figures
\renewcommand{\algorithmicrequire}{\textbf{Input:}}
\renewcommand{\algorithmicensure}{\textbf{Output:}}



\begin{document}


\title{Modeling Behavioral dynamics of Road Networks as Interacting particle systems}
\author{
Sarat Chandra Bobbili\\
%\small{\texttt{baditakumar@iisc.ac.in}}\\
%\large{Indian Institute of Science}
\and
Rooji Jinan\\
%\small{\texttt{ankitdhiman@iisc.ac.in}}\\
%\large{Indian Institute of Science}
\and
Ajay Kumar Badita\\
%\small{\texttt{baditakumar@iisc.ac.in}}\\
%\large{Indian Institute of Science}
}
\maketitle
\section*{Motivation}

Roadway traffic in cities and many other countries may be characterized by lack of lane discipline and a high degree of heterogeneity in the types of vehicles, driving behaviours, roadway geometry, and infrastructure conditions.
However, existing literature in this area is based on rather stylized representations of traffic streams, including assumptions of lane discipline and homogeneity in vehicle types.
These assumptions are difficult to justify and can potentially misguide traffic management practices such as traffic signal control and roadway capacity estimation.
The challenges to be overcome in this context include, among others: $(a)$ realistic representation of vehicular movements and $(b)$ the consideration of various sources of heterogeneity mentioned above.
An equally important challenge is the measurement of traffic streams at both microscopic and macroscopic levels.
In this context,there is a pressing need for accurate modelling, and management of vehicular traffic.


\section*{Background}

Interacting Particle systems have been widely used in applications pertaining to finance, social networks and viral marketing.
Many toy models of stochastic phenomena have been invented since the term is coined in Liggett's 1985 book.
\newline
A brief description of relevant Literature on problems which share common traits from ours are 
\begin{itemize}
\item In \cite{ChengCoRR2017}, the authors characterized the distortion in voting systems when candidates and voters are embedded in a common metric space. 

\item With regards to Social networks, in \cite{romeroIWWWC2016}, the authors studied the effect of external events on the network's change in structure and communications.

\item In \cite{carrilloWS2010}, the authors determined the asymptotic limits of a stochastic model for self propelled interacting particles.
\end{itemize}

\section*{Problem Statement}
Roadway capacity is a key component in constructing and maintaining a road network.
As the number of particles increases, holistic approaches to characterize behavior become increasingly desirable than just tracing the path of one individual in the system. 
In this regard, we would like to identify the factors influencing the stability of the system and hope to solve the following key challenges ubiquitous to any transportation network.

\begin{enumerate}
\item What is the threshold flux of the vehicles beyond which significant distortion takes place in the network?

\item How the averaged quantities like density, travel time, average speed of vehicles on a link change over time?

\item What is the efficient control strategy for the given model to maximize the utilization of the network to its fullest?

\item Under the occurrence of sporadic disruptions like road accidents, what will be the extent of the affected zone?

what should be the ideal control strategy to be used to help the system re-attain the steady state? what is the expected time to reach the steady state?

\end{enumerate} 

\section*{Proposed Solution}

We define the intersections in the road network and vehicles as the two types of particles. The former being a static particle or (Type-1) particle, the latter is a self-propelling particle or (Type-2) particle. Assuming that there is no interaction between particles belonging to the same category, we model the vehicles as self propelling particles.

Under the current model, we wish to employ techniques from \cite{ChengCoRR2017} to characterize the networks evolution.      
%\newpage
\bibliographystyle{IEEEtran}
\bibliography{SPQTproject}


%\section*{References}
%[1] On the Distortion of Voting with Multiple Representative Candidates
%[2] Social Networks under stress
%[3] Self propelled interacting particle systems with roosting force. 
    
\end{document}
