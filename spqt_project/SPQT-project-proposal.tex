\documentclass[a4paper,12pt]{scrartcl}
\usepackage{sectsty}
\subsubsectionfont{\normalfont\itshape}
\usepackage{graphicx}
\usepackage{%
	algorithmic,%
	amsmath,%
	amssymb,%
	amsthm,%
	%booktabs,%
	caption,%
	%fixltx2e,%
	hyperref,%
	%mathpazo,%
	%siunitx,%
	%subcaption,%
	%xspace,%
	pgf,%
	pgfgantt,%
	pgfplots,%
	tikz,%
	url,%
	xcolor,%
	tabto,%
	} 
%\usepackage{color}

\date{}
\usepackage[left=18mm,right=18mm,top=20mm,bottom=20mm]{geometry}
\usepackage[utf8]{inputenc} %-- pour utiliser des accents en français, ou autres
\usepackage[mathscr]{eucal}
%\usepackage[round]{natbib}
%\usepackage[dvipsnames]{xcolor}%
\usepgflibrary{shapes}
\usetikzlibrary{%
	arrows,positioning, shapes.symbols,shapes.callouts,patterns
}
\newcommand{\sS}{\mathscr{S}}
\newcommand{\ie}{i.e.}
\newcommand{\eg}{e.g.}
\newcommand{\reffig}[1]{Figure~\ref{#1}}
\newcommand{\refsec}[1]{Section~\ref{#1}}
\renewcommand{\le}{\leqslant}
\renewcommand{\ge}{\geqslant}
\newcommand{\EQ}[1]{\begin{equation*}#1\end{equation*}}
\setcapindent{1em} %-- for captions of Figures
\renewcommand{\algorithmicrequire}{\textbf{Input:}}
\renewcommand{\algorithmicensure}{\textbf{Output:}}



\begin{document}


\title{Modeling Behavioral dynamics of Road Networks as Interacting particle systems}
\author{
Sarat Chandra Bobbili\\
%\small{\texttt{baditakumar@iisc.ac.in}}\\
%\large{Indian Institute of Science}
\and
Rooji Jinan\\
%\small{\texttt{ankitdhiman@iisc.ac.in}}\\
%\large{Indian Institute of Science}
\and
Ajay Kumar Badita\\
%\small{\texttt{baditakumar@iisc.ac.in}}\\
%\large{Indian Institute of Science}
}
\maketitle
\section*{Motivation}

Roadway traffic in cities and many other countries may be characterized by lack of lane discipline and a high degree of heterogeneity in the types of vehicles, driving behaviours, roadway geometry, and infrastructure conditions.
However, existing literature in this area is based on rather stylized representations of traffic streams, including assumptions of lane discipline and homogeneity in vehicle types.
These assumptions are difficult to justify and can potentially misguide traffic management practices such as traffic signal control and roadway capacity estimation.
The challenges to be overcome in this context include, among others: $(a)$ realistic representation of vehicular movements and $(b)$ the consideration of various sources of heterogeneity mentioned above.
An equally important challenge is the measurement of traffic streams at both microscopic and macroscopic levels.
In this context,there is a pressing need for accurate modelling, and management of vehicular traffic.


\section*{Background}

Interacting Particle systems have been widely used in applications pertaining to finance, social networks and viral marketing.
Many toy models of stochastic phenomena have been invented since the term is coined in Liggett's 1985 book.
\newline
A brief description of relevant Literature on problems which share common traits from ours are 
\begin{itemize}
\item In \cite{ChengCoRR2017}, the authors characterized the distortion in voting systems when candidates and voters are embedded in a common metric space. 

\item With regards to Social networks, in \cite{romeroIWWWC2016}, the authors studied the effect of external events on the network's change in structure and communications.

\item In \cite{carrilloWS2010}, the authors determined the asymptotic limits of a stochastic model for self propelled interacting particles.
\end{itemize}

\section*{Problem Statement}
Roadway capacity is a key component in constructing and maintaining a road network.
As the number of particles increases, holistic approaches to characterize behavior become increasingly desirable than just tracing the path of one individual in the system. 
In this regard, we would like to identify the factors influencing the stability of the system and hope to solve the following key challenges ubiquitous to any transportation network.

\begin{enumerate}
\item What is the threshold flux of the vehicles beyond which significant distortion takes place in the network?

\item How the averaged quantities like density, travel time, average speed of vehicles on a link change over time?

\item What is the efficient control strategy for the given model to maximize the utilization of the network to its fullest?

\item Under the occurrence of sporadic disruptions like road accidents, what will be the extent of the affected zone?

what should be the ideal control strategy to be used to help the system re-attain the steady state? what is the expected time to reach the steady state?

\end{enumerate} 

\section*{Proposed Solution}

We define the intersections in the road network and vehicles as the two types of particles. The former being a static particle or (Type-1) particle, the latter is a self-propelling particle or (Type-2) particle. Assuming that there is no interaction between particles belonging to the same category, we model the vehicles as self propelling particles.

Under the current model, we wish to employ techniques from \cite{ChengCoRR2017} to characterize the networks evolution.      

\section*{System Model}
In Queueing theory,Network of queues have been extensively studied for modeling situations where a customer has to wait in a number of different queues before completing the desired transaction and leaving the system.One such model is the Jackson Networks,wherein we have a network of k interconnected queuing systems which we call nodes.Each of the k nodes receives customers both from outside the network (exogenous inputs) and from nodes within the network (endogenious inputs).\\ 
Drawing a parallel,the key components in any road network intersections play the role of queues and the joining of new vehicles will be identified as customer arrivals.
In a mathematical framework, let $\mathbb{G(V,E)}$ denote the road network, where the $|V|=N+1 , V = I \cup \{0\}$, I = {1,2,...N} is the set of intersections and edges denote the transition probabilities $Q_{ij},\quad i,j\in V$ from one node to another.we will use the term node to discuss intersections to avoid confusion and the rationale behind introduction of a fictitious node $0$ will be explained shortly.\\ 
In a practical setting we have vehicles arriving as exogenous inputs to an arbitrary node, upon service completion joining another node with some probability and continues to traverse for a definite amount of time eventually leaving the network. Notationally, we are regarding the node $0$ as both the source and the sink from which vehicles appear and to which the vehicles disappear.
It is assumed that the exogenous inputs to each node $i\in {1,..N}$ form a Poisson process of rate $r_i$ and that these poisson processes are independent of each other.For simplicity, we view this as a single poisson process of rate $r_0$,with each input independently going to each node i with probability $Q_{0i} = r_i/r_0$.\\

\textit{Lemma:} Each node i contains a single server, and the successive service times at node i are $iid$ with an exponentially service times of rate $\mu_i$.\\
Proof:$< to \quad be \quad filled>$\\
\\
The service times at each node are also independent of the service times at all other nodes and independent of the exogenous arrival times at all nodes.When a vehicle leaves a node $i$, that vehicle is route to node $j$ with probability $Q_{ij}$ and it is also possible for the vehicle to depart from the network entirely which happens with a probability $Q_{i0} = 1 - \sum_{j \in I}Q_{ij}.$ To this end for any vehicle departing from a node $i,$ the next node $j$ is a random variable with pmf\{$Q_{ij},0 \leq j \leq N$\}.

\subsubsection*{key assumptions}
\textit{Assumption 1}: When a vehicle is routed from node $i$ to $j$, it is assumed that the routing is instantaneous; thus at the departure instant from node i,there is a simultaneous endogenous arrival at node j.\\
\textit{Assumption 2}:The time average rate of departures from each node equals the time average rate of arrivals and such rates exist.\\
\textit{Assumption 3}: The combined exogenous and endogenous arrivals as being served under first come first served policy\\
Both assumptions 1 $\&$ 3 are justified because it really makes no difference in which order they are served and how they arrive since the vehicles are statistically identical  simply give a service rate of $\mu_j$ at node $j$. Assumption 2 implies that the queue size does not grow linearly in time. 

\subsubsection*{Markov Structure}
The network as a whole is a Markov process in which the state is a vector $\textbf{n} = {n_1,n_2,....,n_N}$, where $n_i, i \in I$, is the number of vehicles at node $i$. State changes occur either due to exogenous arrivals to nodes or exogenous departures from various nodes, and departures from one node to another node.\\
In a small interval $\delta$ of time given that the state is $\textbf{n}$,\\
\\
If an exogenous arrival at node j occurs in the interval with probability $r_0Q_{0j}\delta$ and state changes to $\textbf{n}^{\prime} = \textbf{n} + \textbf{e}_j,$ where $\textbf{e}_j$ is a vector of appropriate dimension with a one in position $j$.\\
\\
If $n_i > 0 $, an exogenous departure from node $i$ occurs with probability $\mu_iQ_{i0}\delta$ and the state changes to $\textbf{n}^{\prime} = \textbf{n} - \textbf{e}_i.$\\
\\ 
If $n_i > 0 $, a departure from node $i$ entering node $j$ occurs with probability $\mu_iQ_{ij}\delta$ and changes the state to $\textbf{n}^{\prime} = \textbf{n} - \textbf{e}_i + \textbf{e}_j.$ Therefore, the tranisition rates are given by,

\begin{align}
q_{\textbf{n},\textbf{n}^{\prime}}&=r_0Q_{0j}\quad\quad for\enspace\textbf{n}^{\prime} = \textbf{n} + \textbf{e}_j,\quad j\in I\\
&=\mu_iQ_{i0} \quad\quad for\enspace\textbf{n}^{\prime} = \textbf{n} - \textbf{e}_i,\quad n_i > 0, i \in I\\
&=\mu_iQ_{ij} \quad\quad for\enspace\textbf{n}^{\prime} = \textbf{n} - \textbf{e}_i + \textbf{e}_j,\quad n_i > 0, i,j \in I\\
&= 0 \quad\quad\quad\quad otherwise
\end{align}

\subsubsection*{Objective $\mathit{I}$}
We aim to find the steady state probabilities p($\textbf{n}$).  Our approach is to find this by reversibility arugments.\\
Let us define $\lambda_i$ for each $i \in I$, as the time-average overall rate of arrivals to node i, including both exogenous and endogenous arrivals and $\lambda_0 = r_0$.Then these rates must satisfy the equation

\begin{equation}
\lambda_i = \sum_{j=0}^{N} \lambda_j Q_{ji} \quad i \in I
\end{equation}\\
Consider the backward process.Since we have only three kinds of transitions in the forward process.Corresponding to each arrival in the forward process, there is a departure in the backward process; for each forward departure, there is a backward arrival; and for each forward passage from i to j, there is a backward passage from j to i.\\
We make the conjecture that the backward process is itself a Jackson Network with poisson exogenous arrivals at rates $\lambda_0 Q_{0j}^{*}$,service times are exponential with rates   $\mu_i$, and routing probabilities $Q_{ij}^*$.Since each transition from $i$ to $j$ in the forward process must correspond to a transition from $j$ to $i$ in the backward process, we have
\begin{equation}
\lambda_i Q_{ij} = \lambda_j Q_{ji}^* 0 \leq i,j \leq N
\end{equation}
$\lambda_i Q_{ij}$ represents the rate at which forward transitions go from $i$ to $j$, and $\lambda_i$ represnets the rate at which forward transitions leave node $i$ which is also by $\textit{Assumption 2},$ the rate at which forward tranistions enter node $i$.Thus, it is also the rate at which backward transitions leave node $i$.Therefore, we can write the backward transition rates as
\begin{align}
q_{\textbf{n},\textbf{n}^{\prime}}^*&=\lambda_0Q_{0j}^*=\lambda_jQ_{j0}\quad\quad\quad\quad\quad for\enspace\textbf{n}^{\prime} = \textbf{n} + \textbf{e}_j,\quad j\in I\\
&=\mu_iQ_{i0}^*=(\mu_i/\lambda_i)\lambda_0Q_{0i} \quad\quad for\enspace\textbf{n}^{\prime} = \textbf{n} - \textbf{e}_i,\quad n_i > 0, i \in I\\
&=\mu_iQ_{ij}^*=(\mu_i/\lambda_i)\lambda_jQ_{ji} \quad\quad for\enspace\textbf{n}^{\prime} = \textbf{n} - \textbf{e}_i + \textbf{e}_j,\quad n_i > 0, i,j \in I
\end{align}
From Reversibility of a Markov Process we have, the stationary probabilities $p(\textbf{n})$ and the transition rates of  the backward process $q_{\textbf{n}^{\prime},\textbf{n}}^*$ as the solution to the following equations:
\begin{align*}
p(\textbf{n})q_{\textbf{n},\textbf{n}^{\prime}}&=p(\textbf{n}^{\prime})q_{\textbf{n}^{\prime},\textbf{n}}^*\qquad for\enspace all \enspace \textbf{n},\textbf{n}^{\prime};\\
\sum_{\textbf{n}^{\prime}}q_{\textbf{n},\textbf{n}^{\prime}}&=\sum_{\textbf{n}^{\prime}}q_{\textbf{n},\textbf{n}^{\prime}}^*\qquad for\enspace all \enspace \textbf{n}
\end{align*}
Let us define $\rho_i = \lambda_i/\mu_i $,from equations(1)-(3) and (7)-(9), we have
\begin{align*}
p(\textbf{n})&=p(\textbf{n}^{\prime})/\rho_j\quad\quad\quad\quad\quad for\enspace\textbf{n}^{\prime} = \textbf{n} + \textbf{e}_j;\\
p(\textbf{n})&=p(\textbf{n}^{\prime})\rho_i\quad\enspace\quad\quad\quad\quad for\enspace\textbf{n}^{\prime} = \textbf{n} - \textbf{e}_i, n_i > 0;\\
p(\textbf{n})&=p(\textbf{n}^{\prime})\rho_i/\rho_j\quad\quad\quad\quad for\enspace\textbf{n}^{\prime} = \textbf{n} - \textbf{e}_i + \textbf{e}_j, n_i > 0;
\end{align*}
Starting from state p(0,0,...,0) iteratively solving for p(\textbf{n}), we obtain
$p(\textbf{n})=p(0,0,...,0)\prod_{i=1}^N \rho_i^{n_i}$. Since $\sum_{n_1,n_2,...,n_k}p(\textbf{n}) = 1 =\sum_{n_1,n_2,...,n_k} p(0,0,...,0)\prod_{i=1}^N \rho_i^{n_i} = p(0,0,...,0)\sum_{n1} \rho_1^{n_1}\sum_{n2} \rho_2^{n_2}....\sum_{n_N} \rho_N^{n_N}$.\\
Thus we have,\\
\begin{align*}
1&=p(0,0,...,0)(1-\rho_1)^{-1}(1-\rho_2)^{-1}...(1-\rho_N)^{-1}\\
p(0,0,...,0)&=\prod_{i=1}^N(1-\rho_i)\\
p(\textbf{n})&=p(0,0,...,0)\prod_{i=1}^N \rho_i^{n_i}=\prod_{i=1}^N\Big[(1-\rho_i)\rho_i^{n_i}\Big]
\end{align*}
It can be verified the above stationary probabilities satisfy the reversibility conditions.
%\newpage
\bibliographystyle{IEEEtran}
\bibliography{SPQTproject}


%\section*{References}
%[1] On the Distortion of Voting with Multiple Representative Candidates
%[2] Social Networks under stress
%[3] Self propelled interacting particle systems with roosting force. 
    
\end{document}
